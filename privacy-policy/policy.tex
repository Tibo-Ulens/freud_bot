\documentclass[12pt, a4paper]{article}
\usepackage[T1]{fontenc}
\usepackage[utf8]{inputenc}
\usepackage[dutch]{babel}
\usepackage[scaled]{helvet}

\renewcommand{\familydefault}{\sfdefault}

\setlength{\parindent}{0cm}

\begin{document}
	\begin{center}
		\vspace{1cm}
		\huge
		\textbf{FreudBot Privacybeleid}
		\vspace{1cm}
		\normalsize
	\end{center}

	De Discord-server voor de Bachelor of Science in de psychologie aan de
	UGent maakt gebruik van een bot (FreudBot) om te verifiëren of een persoon
	effectief is ingeschreven aan de UGent. De bot bereikt dit door de persoon
	voor hun e-mail te vragen en ze vervolgens een code te versturen via e-mail.
	De persoon zal deze code dan opnieuw aan de bot geven, waarna de bot tot
	slot bevestigt dat deze code overeenkomt met degene die origineel aan de
	persoon werd gegeven.

	Dit privacybeleid wil aantonen aan personen die zich verifiëren via
	FreudBot (hierna te noemen: geverifieerden) hoe deze data verzameld en
	verwerkt worden, en welke rechten de geverifieerden hebben over deze data.

	\section{Wanneer is dit privacybeleid van toepassing?}
	Dit beleid is geldig omtrent alle zaken gerelateerd aan de persoonlijke
	data van geverifieerden die door FreudBot worden bijgehouden.

	FreudBot zal enkel en alleen data over geverifieerden bijhouden, dit begint
	wanneer de persoon in kwestie vrijwillig het '/verify' commando in de
	server gebruikt.

	FreudBot zal geen persoonlijke data opslaan over personen die in de server
	zitten en zich niet verifiëren.

	\section{Welke data worden er verzameld en waarvoor worden ze gebruikt?}
	FreudBot houdt 2 soorten persoonlijke data over geverifieerden bij:
	\begin{itemize}
		\setlength{\itemsep}{0cm}
		\setlength{\parskip}{0cm}

		\item \textbf{E-mailaddressen} - Dit om te verzekeren dat een
		e-mailaddres slechts één keer wordt gebruikt.
		\item \textbf{Discord Account IDs} - Dit om te verzekeren dat een
		zekere Discord account zich slechts één keer verifieert.
	\end{itemize}

	Deze data worden door de geverifieerden aan FreudBot verstrekt op het
	moment dat ze het '/verify' commando gebruiken.

	Deze data worden voor geen enkel doeleinde gebruikt behalve garanderen dat
	een persoon effectief een UGent e-mailaddres heeft.

	\section{Hoe worden de data opgeslagen?}
	Alle data worden opgeslagen in een PostgreSQL databank draaiende op een
	server in het huis van Emiel Verbeeren. Deze databank is niet publiek
	toegankelijk.

	\section{Wie heeft er toegang tot de verzamelde data?}
	Zowel Emiel Verbeeren als Tibo Ulens (zie sectie \ref{contactpersonen})
	hebben toegang tot de verzamelde data.

	Zij beperken hun inzage in de data tot het strikt noodzakelijke om de
	succesvolle operatie van FreudBot te garanderen.

	\section{Welke rechten hebben geverifieerden omstreke hun data?}
	Om hun rechten te laten gelden of om opmerking en/of vragen hierover te
	stellen kunnen geverifieerden contact opnemen met één van de
	contactpersonen vermeldt in sectie \ref{contactpersonen}.

	\vspace{12pt}
	Hieronder volgt een korte oplijsting van de rechten van geverifieerden
	omtrent hun opgeslagen data:

	\begin{itemize}
		\setlength{\itemsep}{0cm}
		\setlength{\parskip}{0cm}

		\item \textbf{Recht op Inzage} - Geverifieerden hebben het recht te
		weten of FreudBot data van hen heeft, welke dit zijn, en op welke
		manier of waarom deze gebruikt worden.
		\item \textbf{Recht op Rectificatie en Aanvulling} - Geverifieerden
		hebben het recht in voorkomend geval foute en/of ontbrekende gegevens
		te laten corrigeren en/of aanvullen.
		\item \textbf{Recht op Vergetelheid} - Geverifieerden hebben het recht
		te vragen dat eenderwelke opgeslagen data van hun worden verwijdert.
		\item \textbf{Recht op Overdraagbaarheid} - Geverifieerden hebben het
		recht hun persoonlijke data op te vragen in een machine-leesbaar,
		gestructureerd document.
		\item \textbf{Recht op Bezwaar} - Geverifieerden hebben het recht
		bezwaar aan te tekenen indien zij vernemen dat hun data onrechtmatig
		verwerkt zouden worden.
	\end{itemize}

	\section{Contactpersonen} \label{contactpersonen}
	\subsection{Emiel Verbeeren}
	Beheerder van de Discordserver en eigenaar van de fysieke server waarop
	FreudBot draait.

	\begin{itemize}
		\setlength{\itemsep}{0cm}
		\setlength{\parskip}{0cm}

		\item \textbf{E-mailaddres} - emiel.verbeeren@ugent.be
		\item \textbf{Discord Gebruikersnaam} - Piglet\#3720
	\end{itemize}

	\subsection{Tibo Ulens}
	Ontwikkelaar van FreudBot, heeft vanop afstand toegang tot de server waarop
	FreudBot draait.

	\begin{itemize}
		\setlength{\itemsep}{0cm}
		\setlength{\parskip}{0cm}

		\item \textbf{E-mailaddres} - tibo.ulens@ugent.be
		\item \textbf{Discord Gebruikersnaam} - Capy\#9030
	\end{itemize}
\end{document}
